\documentclass[dvipdfmx,a4paper,11pt]{article}
\usepackage[utf8]{inputenc}
%\usepackage[dvipdfmx]{hyperref} %リンクを有効にする
\usepackage{url} %同上
\usepackage{amsmath,amssymb} %もちろん
\usepackage{amsfonts,amsthm,mathtools} %もちろん
\usepackage{braket,physics} %あると便利なやつ
\usepackage{bm} %ラプラシアンで使った
\usepackage[top=15truemm,bottom=30truemm,left=25truemm,right=25truemm]{geometry} %余白設定
\usepackage{latexsym} %ごくたまに必要になる
\renewcommand{\kanjifamilydefault}{\gtdefault}
\usepackage{otf} %宗教上の理由でmin10が嫌いなので
\usepackage{showkeys}\renewcommand*{\showkeyslabelformat}[1]{\fbox{\parbox{2cm}{ \normalfont\tiny\sffamily#1\vspace{6mm}}}}
\usepackage[driverfallback=dvipdfm]{hyperref}


\usepackage[all]{xy}
\usepackage{amsthm,amsmath,amssymb,comment}
\usepackage{amsmath}    % \UTF{00E6}\UTF{0095}°\UTF{00E5}\UTF{00AD}\UTF{00A6}\UTF{00E7}\UTF{0094}¨
\usepackage{amssymb}  
\usepackage{color}
\usepackage{amscd}
\usepackage{amsthm}  
\usepackage{wrapfig}
\usepackage{comment}	
\usepackage{graphicx}
\usepackage{setspace}
\usepackage{pxrubrica}
\usepackage{enumitem}
\usepackage{mathrsfs} 

\setstretch{1.2}


\newcommand{\R}{\mathbb{R}}
\newcommand{\Z}{\mathbb{Z}}
\newcommand{\Q}{\mathbb{Q}} 
\newcommand{\N}{\mathbb{N}}
\newcommand{\C}{\mathbb{C}} 
\newcommand{\Sin}{\text{Sin}^{-1}} 
\newcommand{\Cos}{\text{Cos}^{-1}} 
\newcommand{\Tan}{\text{Tan}^{-1}} 
\newcommand{\invsin}{\text{Sin}^{-1}} 
\newcommand{\invcos}{\text{Cos}^{-1}} 
\newcommand{\invtan}{\text{Tan}^{-1}} 
\newcommand{\Area}{\text{Area}}
\newcommand{\vol}{\text{Vol}}
\newcommand{\maru}[1]{\raise0.2ex\hbox{\textcircled{\tiny{#1}}}}
\newcommand{\sgn}{{\rm sgn}}
%\newcommand{\rank}{{\rm rank}}



   %当然のようにやる.
\allowdisplaybreaks[4]
   %もちろん.
%\title{第1回. 多変数の連続写像 (岩井雅崇, 2020/10/06)}
%\author{岩井雅崇}
%\date{2020/10/06}
%ここまで今回の記事関係ない
\usepackage{tcolorbox}
\tcbuselibrary{breakable, skins, theorems}

\theoremstyle{definition}
\newtheorem{thm}{定理}
\newtheorem{lem}[thm]{補題}
\newtheorem{prop}[thm]{命題}
\newtheorem{cor}[thm]{系}
\newtheorem{claim}[thm]{主張}
\newtheorem{dfn}[thm]{定義}
\newtheorem{rem}[thm]{注意}
\newtheorem{exa}[thm]{例}
\newtheorem{conj}[thm]{予想}
\newtheorem{prob}[thm]{問題}
\newtheorem{rema}[thm]{補足}

\DeclareMathOperator{\Ric}{Ric}
\DeclareMathOperator{\Vol}{Vol}
 \newcommand{\pdrv}[2]{\frac{\partial #1}{\partial #2}}
 \newcommand{\drv}[2]{\frac{d #1}{d#2}}
  \newcommand{\ppdrv}[3]{\frac{\partial #1}{\partial #2 \partial #3}}


%ここから本文.
\begin{document}
\pagestyle{empty}



\begin{center}
{\Large 演習問題 2024年6月5日(水)} \\

%\vspace{5pt}
%{ \large 提出締め切り 2024年月日(火) 15時10分00秒 (日本標準時刻)}
\end{center}

%\vspace{2pt}
%\begin{flushleft}
%{ \large \underline{学籍番号: \hspace{4cm} 名前  \hspace{8cm} } }
%\end{flushleft}

\begin{center}
 {\large 下の問題を解け. なお解答は配布した解答用紙に解答すること.}
  \end{center}
 ただし解答に関しては答えのみならず, 答えを導出する過程をきちんと記すこと. 
 また解答用紙は1人1枚以上提出すること.
  
  \vspace{11pt}
 問題1. 次の行列の計算を行え.
 
 \begin{enumerate}
   \setlength{\parskip}{0cm} % 段落間
  \setlength{\itemsep}{0cm} % 項目間
 \item $
 \begin{pmatrix}
 1 &2 \\
 -4&-1\\
  5&-2\\
 \end{pmatrix}
 + 2
 \begin{pmatrix}
 2 &-1 \\
  0&4\\
  -7&0\\
 \end{pmatrix}
 $
 \item $
3 \begin{pmatrix}
 2 &-1&4 \\
 0&3&-5\\
 \end{pmatrix}
 - 2
 \left\{
 \begin{pmatrix}
 0 &1&-2 \\
 7&-5&4\\
 \end{pmatrix}
 - 3
  \begin{pmatrix}
 1 &-2&6 \\
 4&-1&5\\
 \end{pmatrix}
\right\}
 $

 \end{enumerate}
 
\vspace{5pt}
 問題2. 次の行列$A,B,C,D$のうち, 積が定義される全ての組み合わせを求め, その積を計算せよ.
 $$
  A=\begin{pmatrix} %14
 -1 & 2 &-5  \\
 \end{pmatrix} 
 \text{, \,\,} 
B= \begin{pmatrix} %33
 1& 0 & 2\\
 0 & 3 & 0\\
 4 & 0 & 5 \\
 \end{pmatrix} %%32
 \text{, \,\,} 
 C=
  \begin{pmatrix}
 -2 &5 & 3\\
1 &-3&0  \\
 \end{pmatrix}
 \text{, \,\,} 
 D= \begin{pmatrix} %%41
 -4\\
 3 \\
 1
 \end{pmatrix}
 $$
 
 \vspace{5pt}
 問題3.  次の行列を簡約化し, その階数を求めよ.

1.
   \setlength{\parskip}{0cm} % 段落間
  \setlength{\itemsep}{0cm} % 項目間
$
 \begin{pmatrix}
 1& 1& 5  & 0&3\\
 3& 1& 9  & 1&8\\
 2& 0& 4 & 1&5\\
 2& 1& 7 & 1&7\\
 \end{pmatrix}
 $
   \quad 
2. $
 \begin{pmatrix}
 1& 2& 3  & 4&5\\
 2& 3& 4  & 5&6\\
 3& 4& 5 & 6&7\\
 4& 5& 6 & 7&8\\
 5& 6& 7 & 8&9\\
 \end{pmatrix}
 $
    

 
 問題4. 次の連立1次方程式を解け. 
 
 \begin{enumerate}
    \setlength{\parskip}{0cm} % 段落間
  \setlength{\itemsep}{0cm} % 項目間
  \item  $
 \left\{ 
\begin{matrix}
x_1& + &  2x_2&  +& x_3&  = & 0 \\
2x_1& + & 3x_2&  +& x_3&  = & 0 \\
 x_1& + & 2x_2&  +& 2x_3&  = & 0 \\
\end{matrix}
\right.
 $
   \item  $
 \left\{ 
\begin{matrix}
x_1& + &  x_2&  +& 5x_3&  && = & 3 \\
3x_1& + &  x_2&  +& 9x_3& + &x_4& = & 8 \\
2x_1&  &  &  +& 4x_3& + &x_4& = & 5 \\
2x_1& + &  x_2&  +& 7x_3& + &x_4& = & 7 \\
\end{matrix}
\right.
 $
    \item  $
 \left\{ 
\begin{array}{ccccccccccc}
x_1& +& x_2&  -&2x_3	&+&x_4& +&3x_5&=& 1\\
2x_1&-&x_2& + &2x_3&+&2x_4&+&6x_5&= &2 \\
3x_1&+&2x_2& - &4x_3& - &  3x_4  &-&9x_5&= &3\\
\end{array}
\right.
 $
\end{enumerate}

 \newpage
 
  \begin{center}
 {\Large 解答用紙}
% {\Large 演習問題の解答用紙 2024年1月11日(木) } \\
\end{center}

%\vspace{5pt}
%{ \large 提出締め切り 2024年月日(火) 15時10分00秒 (日本標準時刻)}
%\end{center}

%\vspace{2pt}
\begin{flushleft}
{ \large \underline{学籍番号: \hspace{4cm} 名前  \hspace{9cm}   }  }
\end{flushleft}
 

 \end{document}
