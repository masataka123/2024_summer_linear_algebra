\documentclass[dvipdfmx,a4paper,11pt]{article}
\usepackage[utf8]{inputenc}
%\usepackage[dvipdfmx]{hyperref} %リンクを有効にする
\usepackage{url} %同上
\usepackage{amsmath,amssymb} %もちろん
\usepackage{amsfonts,amsthm,mathtools} %もちろん
\usepackage{braket,physics} %あると便利なやつ
\usepackage{bm} %ラプラシアンで使った
\usepackage[top=20truemm,bottom=30truemm,left=25truemm,right=25truemm]{geometry} %余白設定
\usepackage{latexsym} %ごくたまに必要になる
\renewcommand{\kanjifamilydefault}{\gtdefault}
\usepackage{otf} %宗教上の理由でmin10が嫌いなので
\usepackage{showkeys}\renewcommand*{\showkeyslabelformat}[1]{\fbox{\parbox{2cm}{ \normalfont\tiny\sffamily#1\vspace{6mm}}}}
\usepackage[driverfallback=dvipdfm]{hyperref}


\usepackage[all]{xy}
\usepackage{amsthm,amsmath,amssymb,comment}
\usepackage{amsmath}    % \UTF{00E6}\UTF{0095}°\UTF{00E5}\UTF{00AD}\UTF{00A6}\UTF{00E7}\UTF{0094}¨
\usepackage{amssymb}  
\usepackage{color}
\usepackage{amscd}
\usepackage{amsthm}  
\usepackage{wrapfig}
\usepackage{comment}	
\usepackage{graphicx}
\usepackage{setspace}
\usepackage{pxrubrica}
\usepackage{enumitem}
\usepackage{mathrsfs} 

\setstretch{1.2}


\newcommand{\R}{\mathbb{R}}
\newcommand{\Z}{\mathbb{Z}}
\newcommand{\Q}{\mathbb{Q}} 
\newcommand{\N}{\mathbb{N}}
\newcommand{\C}{\mathbb{C}} 
\newcommand{\Sin}{\text{Sin}^{-1}} 
\newcommand{\Cos}{\text{Cos}^{-1}} 
\newcommand{\Tan}{\text{Tan}^{-1}} 
\newcommand{\invsin}{\text{Sin}^{-1}} 
\newcommand{\invcos}{\text{Cos}^{-1}} 
\newcommand{\invtan}{\text{Tan}^{-1}} 
\newcommand{\Area}{\text{Area}}
\newcommand{\vol}{\text{Vol}}
\newcommand{\maru}[1]{\raise0.2ex\hbox{\textcircled{\tiny{#1}}}}
\newcommand{\sgn}{{\rm sgn}}
%\newcommand{\rank}{{\rm rank}}



   %当然のようにやる.
\allowdisplaybreaks[4]
   %もちろん.
%\title{第1回. 多変数の連続写像 (岩井雅崇, 2020/10/06)}
%\author{岩井雅崇}
%\date{2020/10/06}
%ここまで今回の記事関係ない
\usepackage{tcolorbox}
\tcbuselibrary{breakable, skins, theorems}

\theoremstyle{definition}
\newtheorem{thm}{定理}
\newtheorem{lem}[thm]{補題}
\newtheorem{prop}[thm]{命題}
\newtheorem{cor}[thm]{系}
\newtheorem{claim}[thm]{主張}
\newtheorem{dfn}[thm]{定義}
\newtheorem{rem}[thm]{注意}
\newtheorem{exa}[thm]{例}
\newtheorem{conj}[thm]{予想}
\newtheorem{prob}[thm]{問題}
\newtheorem{rema}[thm]{補足}

\DeclareMathOperator{\Ric}{Ric}
\DeclareMathOperator{\Vol}{Vol}
 \newcommand{\pdrv}[2]{\frac{\partial #1}{\partial #2}}
 \newcommand{\drv}[2]{\frac{d #1}{d#2}}
  \newcommand{\ppdrv}[3]{\frac{\partial #1}{\partial #2 \partial #3}}


%ここから本文.
\begin{document}
\pagestyle{empty}



\begin{center}
{\LARGE 期末試験} \\

{2024年度春夏学期 大阪大学 全学共通教育科目 線形代数学I 工(理63〜123)}
\end{center}

\begin{flushright}
 岩井雅崇(いわいまさたか) 2024/07/24
\end{flushright}


%\vspace{2pt}
%\begin{flushleft}
%{ \large \underline{学籍番号: \hspace{4cm} 名前  \hspace{8cm} } }
%\end{flushleft}

%\begin{center}
 %{\large 下の問題を解け. なお解答は配布した解答用紙に解答すること. }
 %{\large 各問題の下に答えを書きこの用紙を提出してください. 問題は両面あります.}
  %\end{center}
  
  \begin{center}
下の問題を解け.  ただし解答に関しては答えのみならず, 答えを導出する過程をきちんと記すこと. 
  \end{center}
  
   \vspace{11pt}
{\Large 第1問.} 
次の行列$A,B,C$のうち, 積が定義される全ての組み合わせを求め, その積を計算せよ.
 $$
  A=\begin{pmatrix} %14
 -2& 1&5 \\
 \end{pmatrix} 
 \text{, \,\,} 
B= \begin{pmatrix} %33
 1& 2 \\
 4 & 3   \\
 \end{pmatrix} %%32
 \text{, \,\,} 
 C=
  \begin{pmatrix}
 -2 &5 \\
1 &-3 \\
3&0  \\
 \end{pmatrix}
 $$
  
  

 
  \vspace{11pt}
{\Large 第2問.}  行基本変形と行列の簡約化を用いて, 次の連立1次方程式の解を求めよ.
 $$
 \left\{ 
\begin{matrix}
 2 x_1&+ &4x_2 &  +& x_3&- &x_4&= & 1\\
x_1&+&2x_2& -&x_3&+&x_4&= & 2\\
2x_1&+ &x_2& + &x_3&+&2x_4&=&-2 \\
 x_1&+ &3x_2& + &2x_3&-&3x_4&=&0 \\
\end{matrix}
\right.
 $$
 %$
% \left\{ 
%\begin{matrix}
%x_1& &         &  +& 2x_3&- &x_4&+ & 2x_5&= 3 \\
%2x_1&+&x_2& + &3x_3&-&x_4&-&x_5&= -1 \\
%-x_1&+&3x_2& - &5x_3&+&4x_4&+&x_5&= -6 \\
%\end{matrix}
%\right.
% $
 
   \vspace{11pt}
{\Large 第3問.} 
%\vspace{11pt}
行列
 % (1).
%$
% \begin{pmatrix}
%1 &2&1 \\
%2 & 1& 3\\
%1&5 &0
% \end{pmatrix}
% $
% (1).
$ \begin{pmatrix}
1&1&1\\
2&3&-1\\
3&4&1 \\
\end{pmatrix}
 $
の逆行列を求めよ. 

  \vspace{11pt}
{\Large 第4問.} 
 次の行列の行列式をそれぞれ求めよ. 
  
 (1). 
 $
 \begin{pmatrix}
1& 1& 1&1&1\\
 1& 2  & 2&2&2\\
1& 2 & 3&3&3\\
1& 2& 3&4&4\\
1& 2& 3&4&5\\
 \end{pmatrix}
 $ 
   (2). 
 $
 \begin{pmatrix}
   3& 0& 1& 0&0\\
  1& 0& 4& 3&0\\
  3& 4& 0  & 2&0\\
  0& 1& 0 & 0&2\\
  6& 0& 2 & 5&0\\
 \end{pmatrix}
  $
 


   \vspace{11pt}
{\Large 第5問.}  連立1次方程式
 $$
 \left\{ 
\begin{matrix}
(\lambda - 3)x_1& -& x_2 &  -& 2x_3&= & 0\\
-x_1&+&(\lambda-3)x_2& +&2x_3&= & 0\\
-2x_1& +&2x_2& + &\lambda x_3&=&0\\
%x_1& +& 3x_2 &  +& 2x_3&+ &4x_4&= & 5\\
%2x_1&+&3x_2& +&x_3&+&3x_4&= & 6\\
%& &6x_2& + &5x_3&+&10x_4&=&a \\
\end{matrix}
\right.
 $$
 が$x_1=x_2=x_3=0$以外の解を持つような$\lambda$の値を全て求めよ.
% の解が存在するような$a$の値を全て求めよ.
 
 
\begin{flushright}
\LARGE{第6問に続く}
\end{flushright}

\newpage 

 \vspace{11pt}
{\Large 第6問}  
 
 次の問題に答えよ. ただし解答に際し授業・教科書で証明を与えた定理に関しては自由に用いて良い. 

 \begin{enumerate}
\renewcommand{\labelenumi}{(\arabic{enumi}).}
% \setlength{\parskip}{0cm} % 段落間
 % \setlength{\itemsep}{0cm}
\item 実数からなる$n$次正方行列$A$について, $^{t}A A=E_n$ならば, $\det (A) = \pm 1$であることを示せ. ここで$E_n$を$n$次の単位行列とし, $^{t}A A$は$^{t}A $と$A$の積のことである.
% \item  実数からなる$2$次正方行列$B$について, $^{t}B B=E_2$かつ$\det (B) =  1$ならば, ある実数$\theta$があって$B =  \begin{pmatrix}\cos \theta& -\sin \theta\\\sin \theta& \cos \theta\\\end{pmatrix}$ と表せることを示せ.
\item 実数からなる$n$次正則行列$A$について, $A$の余因子行列$\tilde{A}$の行列式$\det(\tilde{A})$は$\det(A)^{n-1}$であることを示せ. 
\item 整数からなる$n$次正方行列$A$について, $\det A=1$ならば, 逆行列$A^{-1}$のどの成分も整数であることを示せ. 
\item 1以上の整数からなる100次正方行列$C$で$\det C=-2024$となるものを一つ構成せよ. ただし答えだけでなくその理由も書くこと.
 \end{enumerate} 


%%%%%%%%%%%%%%%%%%%%%%%%%%%%%%%%%%%%%%%%%%%%%%%%%%%%%%%
\begin{comment}

% またこの問題において, $m,n$を正の整数, $E_m$を$m$次の単位行列, $O_{n,m}$を$n\times m$型の零行列とする.
 \begin{enumerate}
\renewcommand{\labelenumi}{(\arabic{enumi}).}
 \setlength{\parskip}{0cm} % 段落間
  \setlength{\itemsep}{0cm}
   % \item $2\times 2$行列$A, B, C$で$ABC\neq CBA$となる例を一つあげよ. 
    \item $2\times 2$行列$A, B$について, $AB$は零行列であるが, $A$も$B$も零行列でない行列$A,B$の例を一つあげよ. 
   % \item $2\times 2$行列$A$について, $A^2=\begin{pmatrix}  1&  0 \\  0 &1 \end{pmatrix}$だが$A$は対角行列でない行列$A$の例を一つあげよ. 
 \item $2\times 2$行列$A, B$について, $AB=\begin{pmatrix}  1&  0 \\  0 &1  \end{pmatrix}$ならば, $AB=BA$であることを示せ.
 \item $2\times 2$行列$A, B, C$について, $\det(ABC) = \det(BAC)$であることを示せ.
   \item $2\times 2$行列$A, B$について, $AB$が正則行列ならば, $A$も$B$も正則行列であることを示せ.
 \end{enumerate} 
 %     \item $2\times 2$行列$A, B$で$AB\neq BA$となる例を一つあげよ. 
  % \item $2\times 2$行列$A$について, $A^2=\begin{pmatrix}  1&  0 \\  0 &1 \end{pmatrix}$だが$A$は対角行列でない行列$A$の例を一つあげよ. 
    % \item $2\times 2$行列$A, B$について, $AB=\begin{pmatrix}  0&  0 \\  0 &0  \end{pmatrix}$であるが, $A$$B \neq \begin{pmatrix}  0&  0 \\  0 &0  \end{pmatrix}$である.
          %\item $2\times 2$行列$A$で. $A$の各成分は実数であるが, $A$の全ての固有値が実数ではない複素数になるものの例をあげよ. 
          %     \item $2\times 2$行列$A$で. $A$の各成分は実数であるが, $A$の全ての固有値が実数ではない複素数になるものの例をあげよ. 


{\Large 第問}  
 \vspace{11pt}
 
   以下の(1)-(5)の各主張について, 正しい場合には証明を与え, 誤っている場合には反例をあげよ.  
   
   
% またこの問題において, $m,n$を正の整数, $E_m$を$m$次の単位行列, $O_{n,m}$を$n\times m$型の零行列とする.
 \begin{enumerate}
\renewcommand{\labelenumi}{(\arabic{enumi}).}
 \setlength{\parskip}{0cm} % 段落間
  \setlength{\itemsep}{0cm}
     \item 行列$A, B$について,$A$と$B$がともに$2\times 2$行列ならば$AB=BA$である. 
 \item $2\times 2$行列$A, B$について, $AB=\begin{pmatrix}  1&  0 \\  0 &1  \end{pmatrix}$ならば, $BA = \begin{pmatrix}  1&  0 \\  0 &1  \end{pmatrix}$である.
 \item $2\times 2$行列$A, B$について, $AB=\begin{pmatrix}  0&  0 \\  0 &0  \end{pmatrix}$ならば, $B = \begin{pmatrix}  0&  0 \\  0 &0  \end{pmatrix}$である.
 \item $2\times 2$行列$A, B, C$について, $\det(ABC) = \det(BAC)$である.
 %\item $n$次正方行列$A$と$n\times m$行列$B$について, $A$が正則行列かつ$AB=O_{n,m}$ならば, $B=O_{n,m}$である.
%  \item $n$次正方行列$A, B$について, $AB=O_{n,n}$ならば, $B=O_{n,n}$である.
  %\item $n$次正方行列$A, B$について, $AB=E_n$ならば, $AB=BA$である.
   \item $2\times 2$行列$A, B$について, $AB$が正則行列ならば, $A$も$B$も正則行列である.
 %\item 「$n$次正方行列$A, B$について, $A$も$B$も正則行列であるならば, $AB$も正則行列である.」
% \item $n$次正方行列$A, B, C$について, $\det(ABC) = \det(BAC)$である.
 %\item 全ての成分が整数である$n$次正則行列$A$について, $\det(A) =\pm1$ならば, 逆行列$A^{-1}$の全ての成分は整数である.
 %\item 全ての成分が整数である$n$次正則行列$A$について, 逆行列$A^{-1}$の全ての成分が整数であるならば, $\det(A) =\pm1$である.

 \end{enumerate} 
 
 ただし次の点に注意すること. 
 \begin{enumerate}
 \renewcommand{\labelenumi}{注意\arabic{enumi}.}
 \setlength{\parskip}{0cm} % 段落間
  \setlength{\itemsep}{0cm}
 \item 授業・教科書で証明を与えた定理に関しては自由に用いて良い. 
 \item 反例とは「ある主張"AならばBである"について, それが成立しない例(つまりAを満たすがBを満たさない例)」のことである. 
例えば「任意の実数$x$について, $x \geqq 0$ならば, $x+1=2$である」という主張は誤りである. その反例として$x=5$が挙げられる. なぜなら$x=5 \geqq0$ではあるが, $x+1 = 5 + 1 =6 \neq 2$であるためである. また$x=1$はこの主張の反例にはならない.
\item 解答に関しては次の解答例を参考にせよ.
   
   \medskip
   (問題例1) $2x =4$ならば$x=2$である.
   
   (解答例1) 正しい. $2x=4$ならば両辺に$\frac{1}{2}$をかけることで
   $x =\frac{1}{2} \cdot 2x = \frac{1}{2}\cdot 4 = 2$となるからである.
   
   \medskip
    (問題例2) $x^2 =4$ならば$x=2$である.
   
   (解答例2) 間違いである. 反例として$x=-2$が挙げられる. $x=-2$とすると$x^2=(-2)^2=4$であるが$x=-2 \neq 2$であるからである.    
 
  \end{enumerate} 
  
  \end{comment}
 %%%%%%%%%%%%%%%%%%%%%%%%%%%%%%%%%%%%%%%%%%%
  
   %%%%%%%%%%%%%%%%%%%%%%%%%%%%%%%%%%%%%%%%%%%
  \begin{comment}


  \newpage

{\Large 第問}  
この問題は行列の対角化の応用を見る問題である. 

フィボナッチ数列$\{a_{n} \}$を
$$
a_{n+2}=a_{n+1} + a_{n} \quad a_1=a_2=1
$$
とする. 
また行列$A$と実数$\alpha, \beta$を次のように定める.
$$
A = \begin{pmatrix}  1& 1  \\  1 &0  \end{pmatrix}
\quad
\alpha = \frac{1 +\sqrt{5}}{2}
\quad
\beta = \frac{1 -\sqrt{5}}{2}
$$
 次の問題に答えよ. 
 \begin{enumerate}
\renewcommand{\labelenumi}{(\arabic{enumi}).}
 \setlength{\parskip}{0cm} % 段落間
  \setlength{\itemsep}{0cm}
  	\item $a_5$を求めよ.
     \item 1以上の自然数$n$について
     $
     \begin{pmatrix}  a_{n+2} \\   a_{n+1} \end{pmatrix}
     = A
     \begin{pmatrix}  a_{n+1} \\   a_{n} \end{pmatrix}
     $
     であることを示せ.
     \item 2以上の自然数$n$について
     $
     \begin{pmatrix}  a_{n} \\   a_{n-1} \end{pmatrix}
     = A^{n-2}
     \begin{pmatrix}  1 \\   1 \end{pmatrix}
     $
     であることを示せ.
     \item $\alpha, \beta$は$A$の固有値であることをしめせ.
     \item 固有値$\alpha,\beta$に関する固有ベクトルをそれぞれ求めよ.
     \item $A^{n}$を求めよ.
     \item (3)と(6)を用いてフィボナッチ数列の一般項$a_{n}$を$\alpha, \beta, n$を用いて表せ. 
 \end{enumerate} 
   \end{comment}
  %%%%%%%%%%%%%%%%%%%%%%%%%%%%%%%%%%%%%%%%%%%
  
 \vspace{11pt}
  {\Large おまけ問題}  
  
  ある部屋に100人集められている. 別の部屋に100個のロッカーがあり, それぞれのロッカーには100人のうち1人の名前が書かれた紙が入れてある. 違うロッカーには違う名前の紙が入っている.

順番を決め, 1人がロッカーの部屋に入り, 50個のロッカーを開ける.
その中に自分の名前があれば、ロッカーを元の状態に戻し, 次の人がロッカー部屋に入り, 50個のロッカーを開ける.
これを繰り返し, 100人が皆自分の名前を見つけられたらゲームは成功. 1人でも名前が見つけられなければゲームは失敗である.

100人はゲームを始める前に議論し戦略を決めることができる. ただしゲームが始まったらお互いに話すこと・情報を伝えることはできない.  このゲームにおいて, 30\%以上の確率で成功する戦略を考えよ.

\vspace{5pt}
[補足]普通にランダムにロッカーを開ける戦略では成功する確率は$(\frac{1}{2})^{100}$となってしまう. 
なおこの問題は100人の囚人問題と呼ばれている. 試験後解答が気になる方は調べてみると良い. 

\begin{flushright}
\LARGE{問題は以上である. }
\end{flushright}
 \end{document}
 
